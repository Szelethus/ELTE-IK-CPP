\documentclass[a4paper,11.5pt]{article}
\usepackage[textwidth=170mm, textheight=230mm, inner=20mm, top=20mm, bottom=30mm]{geometry}
\usepackage[normalem]{ulem}
\usepackage[utf8]{inputenc}
\usepackage[T1]{fontenc}
\PassOptionsToPackage{defaults=hu-min}{magyar.ldf}
\usepackage[magyar]{babel}
\usepackage{amsmath, amsthm,amssymb,paralist,array, ellipsis, graphicx, multirow}
\usepackage[table]{xcolor}
%\usepackage{marvosym}

\usepackage{listings}
\lstset{
	language=C++, 
	basicstyle=\ttfamily, 
	keywordstyle=\color{blue}\ttfamily, 
	stringstyle=\color{red}\ttfamily,
	tabsize = 4
}

\makeatletter
\renewcommand*{\mathellipsis}{%
	\mathinner{%
		\kern\ellipsisbeforegap%
		{\ldotp}\kern\ellipsisgap%
		{\ldotp}\kern\ellipsisgap%
		{\ldotp}\kern\ellipsisaftergap%
	}%
}
\renewcommand*{\dotsb@}{%
	\mathinner{%
		\kern\ellipsisbeforegap%
		{\cdotp}\kern\ellipsisgap%
		{\cdotp}\kern\ellipsisgap%
		{\cdotp}\kern\ellipsisaftergap%
	}%
}
\renewcommand*{\@cdots}{%
	\mathinner{%
		\kern\ellipsisbeforegap%
		{\cdotp}\kern\ellipsisgap%
		{\cdotp}\kern\ellipsisgap%
		{\cdotp}\kern\ellipsisaftergap%
	}%
}
\renewcommand*{\ellipsis@default}{%
	\ellipsis@before
	\kern\ellipsisbeforegap
	.\kern\ellipsisgap
	.\kern\ellipsisgap
	.\kern\ellipsisgap
	\ellipsis@after\relax}
\renewcommand*{\ellipsis@centered}{%
	\ellipsis@before
	\kern\ellipsisbeforegap
	.\kern\ellipsisgap
	.\kern\ellipsisgap
	.\kern\ellipsisaftergap
	\ellipsis@after\relax}
\AtBeginDocument{%
	\DeclareRobustCommand*{\dots}{%
		\ifmmode\@xp\mdots@\else\@xp\textellipsis\fi}}
\def\ellipsisgap{.1em}
\def\ellipsisbeforegap{.05em}
\def\ellipsisaftergap{.05em}
\makeatother

\usepackage{hyperref}
\hypersetup{
	colorlinks = true	
}

\begin{document}
	%%%%%%%%%%%RÖVIDÍTÉSEK%%%%%%%%%%
	\setlength\parindent{0pt}
	\def\s{\hspace{0.2mm}\vphantom{\beta}}
	\def\Z{\mathbb{Z}}
	\def\Q{\mathbb{Q}}
	\def\R{\mathbb{R}}
	\def\C{\mathbb{C}}
	\def\N{\mathbb{N}}
	\def\Ra{\overline{\mathbb{R}}}
	
	\def\sume{\displaystyle\sum_{n=1}^{+\infty}}
	\def\sumn{\displaystyle\sum_{n=0}^{+\infty}}
	
	\def\narrow{\underset{n\rightarrow+\infty}{\longrightarrow}}
	\def\limn{\displaystyle\lim_{n\to +\infty}}
	\def\limx{\displaystyle\lim_{x\to +\infty}}
	
	\theoremstyle{definition}
	\newtheorem{theorem}{Tétel}[subsection] 
	
	\theoremstyle{definition}
	\newtheorem{definition}[theorem]{Definíció} 
	\newtheorem{example}[theorem]{Példa} 
	\newtheorem{task}[theorem]{Feladat} 
	\newtheorem{note}[theorem]{Megjegyzés}
	%%%%%%%%%%%%%%%%%%%%%%%%%%%%%%%%%%%%%%%%%%%%%%%%%%%%%%%%%%%%%%%%%%%%%
	\begin{center}
		{\LARGE\textbf{C++}}
		
		{\Large Gyakorlat jegyzet}
		
		12. óra.
	\end{center}
	A jegyzetet \textsc{Umann} Kristóf készítette \textsc{Brunner} Tibor  előadásán. (\today)
	\medskip
	
	
	%TODO 023 beugro pdf
	%TODO magyarázatok
	\section{Beugró kérdések}
	EZek sose fordulhatnak már elő.
	\begin{enumerate}
		\item Implementációfüggő
		\item \texttt{bool == char}
		\item 10 (8as számrendszer miatt)
		\item \texttt{\#elseif}
		\item \texttt{public}, ugyanis a register azt jelenti, hogy a változót a \texttt{register}be tárolja, azonabn ezt jobban tudja a fordító, így nem érdemes kiírni. az \texttt{auto} azt jelenti hogy a változó stacken legyen, ez régebben implicit módon mindehol ott volt, de mostmár nem, sőt, c++11ben msát is jelent.
		\item 5
		\item Az első az igaz, továbbá sok egyéb dolgot is generál még.
		\item nem definiált, mert y még nem volt inicializálva.
		\item \texttt{std::list}
		\item 1
		\item AZ objektumnak csak a konstans tagfüggvényei hívhatók meg
		\item polimorf osztályra van szükség, mert csak mutatókon tudjuk használni. Ha referencián használnánk, és null-t adna vissza, akkor kivételt dob, mert a referencia nem leht null
		\item második
		\item harmadik a jó, mert lehet egy függvényen kívüli függvény, és az uccsó szintaktikusan rosszz.
		\item utolsó.
	\end{enumerate}
	\section{Mintaviszga megoldás}
	Írjunk header fájlt, mely ezzel jól fordul:
\begin{lstlisting}
#include <iostream>
#include "lineedit.h"
#include <string>
#include <algorithm>
#include <iterator>
#include "lineedit.h"
#include <list>
#include <vector>

const int max = 1000;

int main()
{
	int your_mark = 1;
	
	//2-es
	
	line_editor<std::vector<int>, int> lev;
	for( int i = 0; i < max; ++i )
	{
		lev.press( i );
		lev.home();
		lev.press( i );
	}
	std::vector<int> v = lev.enter();
	
	line_editor<std::list<double>, double> lel;
	lel.press( 4.8 );
	lel.home();
	lel.press( 1.1 );
	std::list<double> c = lel.enter();
	
	line_editor<std::string, char> les;
	les.press( 'W' );
	les.press( 'o' );
	les.press( 'r' );
	les.press( 'l' );
	les.press( 'd' );
	les.home();
	les.press( 'H' );
	les.press( 'e' );
	les.press( 'l' );
	les.press( 'l' );
	les.press( 'o' );
	
	std::string s = les.enter();
	if ( "HelloWorld" == s && "" == les.enter() && 2.2 > *(c.begin()) &&
	2 * max == v.size() && max - 1 == v[ 0 ] )
	{
		your_mark = c.size();
	}
	
	//3-as
	
	les.press( 'H' );
	les.press( 'e' );
	les.press( 'l' );
	les.press( 'l' );
	les.press( 'o' );
	les.home();
	les.insert();
	les.press( 'H' );
	les.press( 'a' );
	s = les.enter();
	
	
	for( int i = 0; i < max; ++i )
	{
		lev.press( 2 );
	}
	lev.home();
	lev.insert();
	for( int i = 0; i < max; ++i )
	{
		lev.press( 1 );
	}
	v = lev.enter();
	
	lel.press( 7.9 );
	lel.press( 1.2 );
	lel.home();
	lel.insert();
	lel.press( 1.5 );
	lel.insert();
	lel.press( 3.7 );
	c = lel.enter();
	
	if ( 1.7 > c.front() && 1.3 < c.front() && v[ max / 2 ] == v[ max / 5 ] &&
	1.4 > c.back() && 1U * max == v.size() && "Hallo" == s && 1 == v[ max / 4 ] )
	{
		your_mark = c.size();
	}
	
	//4-es
	
	line_editor<std::vector<int> > llev;
	llev.press( 3.3 );
	llev.press( 1.1 );
	llev.home();
	llev.del();
	std::vector<int> lv = llev.enter();
	
	line_editor<std::string> lles;
	lles.press( 'J' );
	lles.press( 'a' );
	lles.press( 'v' );
	lles.press( 'a' );
	lles.backspace();
	lles.backspace();
	std::string f = lles.enter();
	
	if ( "Ja" == f && lv[ 0 ] < 1.3 )
	{
		your_mark += lv.size();
	}
	
	//5-os
	
	line_editor<std::list<int> > lle;
	lle.press( 3 );
	lle.home();
	lle.insert();
	
	llev.press( 8 );
	llev.press( 2 );
	llev.home();
	
	lle.swap( llev );
	lle.press( 1 );
	std::list<int> cl = lle.enter();
	v = llev.enter();
	if ( 1 == cl.front() && cl.size() > v.size() && 3 == v[ 0 ] )
	{
		your_mark += v.size();
	}
	
	
	std::cout << "Your mark is " << your_mark;
	std::endl( std::cout );
}
\end{lstlisting}

	Én most kiszedtem a kommenteket, de minden blokk (tehát pl. a 2-es és 3as közötti) ki van kommentezve.
	
	A feladat az, hogy egy billentyűzetet szimuláljon. TUdjuk hogy \texttt{line\_editor} a konténer neve, és látjuk, hogy 2 template paraméterrel rendelkezik. Ezen kívül van egy \texttt{press} függvény, mely egy billentyt rak be, \texttt{home()} a sor elejére ugrik, és \texttt{enter} új sort kezd.
	\subsection{2-es}
	Hozzuk létre a szükséges header fájlt, melyet MINDIG a header guard-dal kezdünk. Lévén az osztály neve és template paraméterei adottak:
	\begin{lstlisting}
#ifndef LINE_EDIT_H
#define LINE_EDIT_H

template <typename Cont, typename CharT>
class line_editor
{

};

#endif 
	\end{lstlisting}
	Kéne egy konténert választani, melyben eltároljuk a karaktereket. Itt logikus választás az \texttt{std::list}, mert gyakran kell majd szúrnunk a sor közepébe.
	\begin{note}
		Hazsnálhatnánk vectort is, sőt bármit, melyet védéskor kellőképpen meg tudunk indokolni.
	\end{note}
\begin{lstlisting}
#include <list>

template <typename Cont, typename CharT>
class line_editor
{
	std::list<CharT> line;
};
\end{lstlisting}
	A következő kérdés, hogyan reprezentáljuk a kurzort, mely jelzi hova kell beszúrni. Erre praktikus megoldás lehet egy iterátor használata.

\begin{lstlisting}
template <typename Cont, typename CharT>
class line_editor
{
	std::list<CharT> line;
	typename std::list<CharT>::iterator cursor;
};
\end{lstlisting}

	Írjuk fel a szükséges függvényeket.
	
\begin{lstlisting}
template <typename Cont, typename CharT>
class line_editor
{
	std::list<CharT> line;
	typename std::list<CharT>::iterator cursor;
public:
	line_editor() : cursor(line.end()) {}
	
	void press(CharT c)
	{
	}
	
	void home()
	{
		cursor = line.begin();
	}
	Cont enter()
	{
	}
};
\end{lstlisting}
	Mivel az iterátor könnyen invalidálódhat ha új elemt szúrunk be, így az \texttt{insert} az \texttt{std::list}-nél visszaad egy iterátort az új elemre, így érdemes a \texttt{cursor}-t erre beállítani, és egyel előrébb vinni, hgoy a sor végén leghyünk.
	\begin{lstlisting}
void press(CharT c)
{
	cursor = line.insert(cursor, c);
	cursor++;
}
	\end{lstlisting}
	Az enternél vissza kell adnunk egy, a tempalte paraméterként megadott konténer típususával megyező kontért, mert tartalmazza az eddig begépelt elemeket, majd törli a listát. Mivel minden STL konténer rendelkezik olyan konstruktorral, mely egy iterátor párt vár, és ez alapján a két iterátor közti elemeket beszúrja, így ezt bátran írhatjuk:
	\begin{lstlisting}
Cont enter()
{
	Cont tmp(line.begin(), line.end());
	line.clear();
	cursor = line.end();
	return tmp;
}
	\end{lstlisting}
	%TODO google dat

	Azt, hgoy mi szerint (érték, referencia) veszünk át, szintén meg kell indokolni.
	
	Ez a típus reguláris, így nem kell egyéb függvényeket megírnom.
	%TODO példál
	
	\subsection{3-as}
	Szükségünk lesz egy \texttt{insert} metódusra, mely ki be kapcsolja az \texttt{insert} billentyűt.
\begin{lstlisting}
template <typename Cont, typename CharT>
class line_editor
{
	std::list<CharT> line;
	typename std::list<CharT>::iterator cursor;
	
	bool isInsert;
public:
	void insert()
	{
		isInsert = !isInsert;
	}
	void press(CharT c)
	{
		if(isInsert && cursor != line.end())
			*cursor = c;
		else
			cursor = line.insert(cursor, c);
		cursor++;
	}
	//...
};
\end{lstlisting}
	\subsection{4-es}
	Megfigyelhetjük, hogy itt már csak 1 template paraméter kell. EZ azt jelenti, hogy default template arguemntumotkéne alklamaznunk, mely megegyezik a konténer által tárolt típussal.
\begin{lstlisting}
template <typename Cont, typename CharT = Cont::value_type>
class line_editor
{
	//...
};
\end{lstlisting}
	 Mivel guglizni lehet, nem kell megijedni, ha nem tudjuk mit kéne csinálni. ELőfordulhat olyan, hogy olyan dologgal tlaálkozunk szembe, amit se EA-n, se GY-n nem mondtak el, ez azért van ,mert ennek a tárgynak vélja az, hogy ezekkel is meg tudjunk közdeni.
	 
	Ezen kívül két új metódust is meg kéne írnunk.
\begin{lstlisting}
template <typename Cont, typename CharT = Cont::value_type>
class line_editor
{
	//...
public:	
	void backspace() //kitorol egy karaktert
	{
		cursor--;
		del();
	}
	void del() //kovetkezo elem torlese
	{
		cursor = line.erase(cursor);
	}
	//...
};
\end{lstlisting}
	\subsection{5-ös}
	Meg kell írnunk egy swap függvényt, mely 2, különböző template paraméterrel rendelkező line\_editort megcserél.
	\begin{lstlisting}
#include <algorithm>
	
template <class T, class U>
void swap(line_editor<T,U> &le)
{
	std::swap(line, le.line);
	std::swap(cursor, le.cursor);
	std::swap(isInsert, le.isInsert);
}
	\end{lstlisting}
	Itt hibát kapunk, mivel ezek más típusok, így private-ok az adattagok egymás felé.
	\begin{lstlisting}
template <class T, class U>
friend class line_editor;
	\end{lstlisting}
	Ezzel elértük, hogy bármilyen template paraméterre, minden line editor haver legyen
%TODO linkek
\end{document}
