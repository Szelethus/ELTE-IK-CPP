\documentclass[a4paper,11.5pt]{article}
\usepackage[textwidth=170mm, textheight=230mm, inner=20mm, top=20mm, bottom=30mm]{geometry}
\usepackage[normalem]{ulem}
\usepackage[utf8]{inputenc}
\usepackage[T1]{fontenc}
\PassOptionsToPackage{defaults=hu-min}{magyar.ldf}
\usepackage[magyar]{babel}
\usepackage{amsmath, xcolor, amsthm,amssymb,paralist,array, ellipsis, graphicx}
%\usepackage{marvosym}

\usepackage{listings}
\lstset{language=C++, basicstyle=\ttfamily, keywordstyle=\color{blue}\ttfamily, stringstyle=\color{red}\ttfamily}

\makeatletter
\renewcommand*{\mathellipsis}{%
	\mathinner{%
		\kern\ellipsisbeforegap%
		{\ldotp}\kern\ellipsisgap%
		{\ldotp}\kern\ellipsisgap%
		{\ldotp}\kern\ellipsisaftergap%
	}%
}
\renewcommand*{\dotsb@}{%
	\mathinner{%
		\kern\ellipsisbeforegap%
		{\cdotp}\kern\ellipsisgap%
		{\cdotp}\kern\ellipsisgap%
		{\cdotp}\kern\ellipsisaftergap%
	}%
}
\renewcommand*{\@cdots}{%
	\mathinner{%
		\kern\ellipsisbeforegap%
		{\cdotp}\kern\ellipsisgap%
		{\cdotp}\kern\ellipsisgap%
		{\cdotp}\kern\ellipsisaftergap%
	}%
}
\renewcommand*{\ellipsis@default}{%
	\ellipsis@before
	\kern\ellipsisbeforegap
	.\kern\ellipsisgap
	.\kern\ellipsisgap
	.\kern\ellipsisgap
	\ellipsis@after\relax}
\renewcommand*{\ellipsis@centered}{%
	\ellipsis@before
	\kern\ellipsisbeforegap
	.\kern\ellipsisgap
	.\kern\ellipsisgap
	.\kern\ellipsisaftergap
	\ellipsis@after\relax}
\AtBeginDocument{%
	\DeclareRobustCommand*{\dots}{%
		\ifmmode\@xp\mdots@\else\@xp\textellipsis\fi}}
\def\ellipsisgap{.1em}
\def\ellipsisbeforegap{.05em}
\def\ellipsisaftergap{.05em}
\makeatother

\usepackage{hyperref}

\begin{document}
	%%%%%%%%%%%RÖVIDÍTÉSEK%%%%%%%%%%
	\setlength\parindent{0pt}
	\def\s{\hspace{0.2mm}\vphantom{\beta}}
	\def\Z{\mathbb{Z}}
	\def\Q{\mathbb{Q}}
	\def\R{\mathbb{R}}
	\def\C{\mathbb{C}}
	\def\N{\mathbb{N}}
	\def\Ra{\overline{\mathbb{R}}}
	
	\def\sume{\displaystyle\sum_{n=1}^{+\infty}}
	\def\sumn{\displaystyle\sum_{n=0}^{+\infty}}
	
	\def\narrow{\underset{n\rightarrow+\infty}{\longrightarrow}}
	\def\limn{\displaystyle\lim_{n\to +\infty}}
	\def\limx{\displaystyle\lim_{x\to +\infty}}
	
	\theoremstyle{definition}
	\newtheorem{theorem}{Tétel}[subsection] 
	
	\theoremstyle{definition}
	\newtheorem{definition}[theorem]{Definíció} 
	\newtheorem{example}[theorem]{Példa} 
	\newtheorem{task}[theorem]{Feladat} 
	\newtheorem{note}[theorem]{Megjegyzés}
	%%%%%%%%%%%%%%%%%%%%%%%%%%%%%%%%%%%%%%%%%%%%%%%%%%%%%%%%%%%%%%%%%%%%%
	\begin{center}
		{\LARGE\textbf{C++}}
		
		{\Large Gyakorlat jegyzet}
		
		1 óra.
	\end{center}
	A jegyzetet \textsc{Umann} Kristóf készítette \textsc{Horváth} Gábor  előadásán. (\today)
	\section{Bevezető.}
	A c++ egy nagyon hatékony kódot tud alkotni. Akkor amikor a facebooknak a kódját 1\%ot sikerült javítani, óriási összegeket spóroltak meg áramköltségen. Nem használ garbage collectort: az olyan nyelveknél, amiben van, néha addig szünetelteti a programot, amíg le nem fut. Itt ilyen probléma sosincs.
	
	\medskip
	Cél: Úgy leadni a tárgyat hogy ne fussunk nagyon alap hibákba, és kialakítsanak valami intuíciót, mert mindent amit hallottunk a gépekkel/programozással kapcsolatban, az hazugság volt.
	\subsection{Mi az hogy c++?}
	Egy olyan szabvány, ami megmondja melyik kulcsszó mit jelentsen, és hogyan forduljon le a program. Az első ilyen szabvány a {c++98} volt. További szabványai: {c++03}, {c++11}, {c++14}, {c++17}.
	
	\medskip
	Különböző fordítók is vannak: MSVC (visual studio), GCC, CLANG. 
	\section{Első óra.}
	\subsection{Nem definiált viselkedések.}
	\begin{example}\ 
		
		\begin{lstlisting}
int main()
{
	int i = 0;
	std::cout << ++i << ++i << std::endl;
}
		\end{lstlisting}
		
		Lefordítás és futtatás után különböző eredményeket kaphatunk, mert itt az mikor értékelődik ki a \texttt{++i}, az \textbf{nem definiált.} Ha a szabvány nem terjed ki arra, hogy mit tegyen a fordító, akkor teljesen a fordítótól függ, hogy az hogyan lesz megoldva. 
		
		\medskip
		Gyakran eldönthetetlen is hogy melyik a jobb megoldás.
		
		\medskip
		Ez lehetőséget ad a fordítónak arra hogy \textbf{optimalizáljon.} 
	\end{example}
	
	A c++ban nincs megadva hogy az utasításokank milyen sorrendben kell végrehajtódniuk.
	
	\medskip
	A c++ban van un. szekvenciapontok, és a szabvány csak at mondja ki, hogy egy szekvenciapont előtti kód előbb hajtódjon végre mint az utánalevő.  Mivel itt az \textit{i} értékadása után és csak az \texttt{std::endl} után van szekvenciapont, így az, hogy milyen sorrendben legyen kiértékelés, az teljesen a fordítóra van bízva.
	
	\medskip
	Az, hogy mind a két utasítás szekvenciapont elválasztása nélkül ugyanazt a memóriaterületet akarja müdosítani, \textbf{nem definiált} viselkedést eredményez.
	\begin{note}
		Az a program, amely nem definiált viselkedéseket tartalmaz, nem számít jó programnak.
	\end{note}
	\subsection{Nem specifikált viselkedések.}
	Itt most több dolog között várhat, hogy először értékel növel az egyiknek, értéket növel a másiknak, stb\ldots Az, hogy itt melyiket válassza, az \textbf{nem specifikált}.
	
	A nem specifikált viselkedés önmagában nem probléma, mert ha módosulna a kód erre:
	
	\begin{example}\ 
		
		\begin{lstlisting}
int main()
{
	int i = 0;
	int j = 0;
	std::cout << ++i << ++j << std::endl;
}
		\end{lstlisting}
	\end{example}
	Itt nincs gond, mert azt továbbra se tudjuk, hogy \texttt{i} vagy \texttt{j} mikor értékelődik ki, (\textit{nem specifikált}), de azt hogy 11-et fog kiírni, azt biztosra tudjuk (az hogy mi lesz kiírva a konzolja, az \textit{definiált}).
	
	\section{A fordító működése.}
	
	\begin{center}
		$\begin{matrix}
		&\fbox{.h}\\
		&\swarrow\searrow&\\
		\fbox{\text{.cpp}}&&\fbox{.cpp}\\
		\downarrow&&\downarrow\\
		\fbox{CU}&&\fbox{CU}\\
		\downarrow&&\downarrow\\
		\fbox{O}&&\fbox{O}\\
		&\searrow\swarrow&\\
		&\fbox{exe}&
	\end{matrix}$
	\end{center}
	\begin{compactitem}
		\item Preprocesszálás
		\item Fordítás (tárgy kód létrehozása)
		\item linkelés
	\end{compactitem}
	\subsection{Preprocesszor.}
	Amikor csak lehet, kerüljük el, és ha lehet, csak include-oláshoz használjuk. A preprocesszor \textbf{buta.} A preprocesszor ezen kívül \textbf{nyelvfüggetlen.}
	\begin{example}\ 
		alma.h
		\begin{lstlisting}
#define ALMA 5

ALMA ALMA ALMA
		\end{lstlisting}
		Command line-ban preprocesszálási fázis eljuthatunk a \texttt{cpp alma.h} kód segítségével.
		
		Ezután azt látjuk hogy a fájban annyi van hogy \texttt{5 5 5}.
	\end{example}
	\begin{example}\
		
		\begin{lstlisting}
#define ALMA 5
#define KORTA
ALMA ALMA ALMA

#ifdef KORTE
MEGVAN
#else
KORTE
#endif
		\end{lstlisting}
		Kimenetelben csak az van benne, hogy MEGVAN.
	\end{example}
	
	\begin{example}\
		
		\begin{lstlisting}
#define ALMA 5
#define KORTA
ALMA ALMA ALMA

#undef KORTE

#ifdef KORTE
MEGVAN
#else
KORTE
#endif
		\end{lstlisting}
		Itt a kimenetelben KORTE lesz.
	\end{example}
	A preprocesszor segít abban is, hogy bár össze-visza vágja a kódot, de észben tartja, hogy melyik sor volt eredetileg ami módosult.
	\begin{note}\
		\texttt{g++ -save-temps hello.cpp} $\rightarrow$ Minden köztes állapot meglegyen.
	\end{note}
	Mivel az \texttt{include}-ok szó szerint beillesztenek, nagyon gyorsan meg tudják dobni a kód méretét, ami a fordítást lassítja, Ezért óvatosan kell vele bánni.
	\begin{example}\ 
		pp.h
		\begin{lstlisting}
#define ALMA 5
#define KORTA
#include "pp.h"
ALMA ALMA ALMA

#undef KORTE

#ifdef KORTE
MEGVAN
#else
KORTE
#endif
		\end{lstlisting}
	\end{example}
		
	Ilyenkor a fordító szól, hogy túl mélyre include-olt egy idő után, és leállítja a preprocesszálást.
	
	Itt könnyű lábon lőni magunkat, mert nagyon sok halmozott include esetén nehéz átlátni hogy mi mit akar bevágni.
	\begin{note}
		\texttt{\#define KORTE$\backslash$ALMA} azt jelenti, hogy \texttt{KORTE}-t \texttt{ALMA}-ra cseréli.
	\end{note}
	\begin{example}\ 
		
		pp.h
		\begin{lstlisting}
#ifndef _PP_H_
#define _PP_H_

#include "pp.h"
FECSKE

#endif
		\end{lstlisting}
		
		alma.h
		\begin{lstlisting}
#define ALMA 5
#define KORTA
#include "pp.h"
#include "pp.h"
#include "pp.h"
#include "pp.h"
#include "pp.h"
ALMA ALMA ALMA

#undef KORTE

#ifdef KORTE
MEGVAN
#else
KORTE
#endif
		\end{lstlisting}
	\end{example}
	
	Ezzel a trükkel megmentettük magunkat attól, hogy többször be legyen illesztve \texttt{FECSKE}. Itt először megnézzük hogy \texttt{\_PP\_H\_} van-e definiálva. Ha nincs, definiáljuk, és beillesztünk. Ekkor legközelebb ha megint ezt meg akarnánk tenni, nem illeszti be \texttt{FECSKE}-t, mert \texttt{\#ifndef \_PP\_H\_} megállítja.
	
	Ez az úgy nevezett \textbf{header guard.}
	
	\medskip
	A preprocesszort sokkal többet tud, de gyakran nem érdemes használni.
	\subsection{Fordítás.}
	\begin{example}\
		
		\begin{lstlisting}
int factorial(int n)
{
	if (n <= 0) return 1;
	else return n*factorial(n-1);
}

int main()
{
	std::cout << factorial(5) << std::endl;
}
		\end{lstlisting}
		
		Itt assembly kód vizsgálatával megállapítható, hogy 5-el lett meghívva a faktoriális függvény.
		\begin{note}
			\texttt{g++ -save-temps hello.cpp -O2}
			
			Kevésbé aggresszív optimalizálás.
		\end{note}
		
		Ilyenkor assembly kódból kiolvasható, hogy a kód (ezen az implementáción) úgy fordult, hogy rögtön kiszámolta, és 120at írt be egyből.
		
		A fordító sokat \textbf{optimalizál.}
	\end{example} 
	\begin{note}
		\texttt{-O3} Olyan optimalizálásokat is tartalmazhat, amik ronthatnak a kódon, vagy nem eredményezheti ugyanazt.
		
		\texttt{-O2} Kevésbé aggresszív, és ha optimalizál is, nem változtatja meg, a kód jelentését.
		
		Mivel nem definiált viselkedésekre extrém rosszul tud reagálni az \texttt{-O3}, így nem érdemes túl gakran használni.
	\end{note}
	
	\subsection{Linkelés.}
	\begin{example}\ 
		
		fecske.cpp
		\begin{lstlisting}
void fecske()
{

}
		\end{lstlisting}
		
		main.cpp
		\begin{lstlisting}
int factorial(int n)
{
	if (n <= 0) return 1;
	else return n*factorial(n-1);
}

int main()
{
	fecske();
	std::cout << factorial(5) << std::endl;
}
		\end{lstlisting}
		
		Ez nem fog fordulni, mert vagy csak a fecske.h-t, vagy a fecske.h-t látja, és nem látjuk őket egyszerre. Megoldás az ha forward deklarálunk, \texttt{void fecske()}-t beillesztjük a main függvény felé. Ekkor a fordítás a linkelés fázisánál hibát jlez, mert nem találja a fecske függvény definícióját.
		
		Ha lefordítom külön a main.cpp-t, a fecske.cpp-t, akkor kapok két object fájlt, és a
		
		\texttt{g++ main.o fecske.o}
		
		fordítja egybe. Rövidebb, ha egyből a cpp fájlokat rakjuk össze.
	\end{example}
	Ha a fecske.cpp-ben sokminden van, akkor nem célszerű egyesével forward deklarálni.Ennél egyszerűbb egy ehader fájl megírása, amiben definiáljuk a fecske.cpp függvényeit.
	
	fecske.h
	\begin{lstlisting}
#ifndef _FECSKE_H_
#define _FECSKE_H_
void fecske();
#endif
	\end{lstlisting}
	Ilyenkor elég a \texttt{fecske.h}-t includeolni.
	
	\begin{definition}
		Definíció: Amikor megmondjuk a függvénynek hogy mit csináljon.
	\end{definition}
	\begin{definition}
		Deklaráció: Amikor csak szólunk a compiler-nek, hogy mit várjon egy függvénytől. (pl. mivel tér vissza, mik a paraméterei egy adott függvénynek)
	\end{definition}
	\begin{note}
		Ezek a definíciók csak a függvényekre vonatkoznak, de nem csak függvényeken értelmezzük őket. Erről később lesz részletesebb leírás.
	\end{note}
	Szokás a fecske.h-t a fecske.cpp-be is beilleszteni, mert ha hibát követünk el és véletlenül rosszul definiálnék a fordítási egységben, akkor kapnék hibaüzenetet.
	\begin{definition}
		One Definition Rule (ODR): Bármit lehet bármennyiszer deklarálni, azonban ha azok ellentmondanak egymásnak, akkor fordítási hibát kapunk.
	\end{definition}
	\begin{example}
		
		fecske.h
		\begin{lstlisting}
#ifndef _FECSKE_H_
#define _FECSKE_H_
void fecske();
int macska() {}
#endif
		\end{lstlisting}
		
		Itt ha több fordítási egységet fordítok egybe, akkor a preprocesszor több macska függvényt definíciót csinál, és linkeléskor a linker azt látja, hogy egy függvény többször van definiálva, és fordítási hibát eredményez.
	\end{example}
	\begin{note}
		A header fájlokban NAGYON nem érdemes definíciókat rakni (bár kivétel létezhet, pl. template-ek, de általában elkerülendő).
	\end{note}
	\subsection{Implementáció által definiált dolgok.}
	A szabvány nem köti meg, hogy pl. egy \textit{int} egy adott platformon mennyi byte-ból álljon. Ez állandó, egy adott platformon egy adott fordító mindig ugyanakkorát hoz létre, de platform/fordítóváltás esetén ez változhat.
	\section{Amiben hazudtak.}
	\begin{example}
		A main függvény végrehajtásával kezdődik a program futása tényleg?
		\begin{lstlisting}
std::ostream& o = std::cout << "Hello";
int main()
{
	std::cout << "valami";
}
		\end{lstlisting}
		Kimenet: \texttt{Hellovalami}.
		
		Tehát ez nem volt igaz. A program végrehajtásánál az első lépés a globális változók inicializálása. Ennek az oka az, hogyha \texttt{o}-t akarnám használni a main() függvény első sorában, akkor tudjam (lévén elvárás az, hogy egy globális változó bárhol elérhető legyen).
		\begin{lstlisting}
int f()
{
	return 5;
}
int x = f();

int main()
{
	std::cout << "valami";
}
		\end{lstlisting}
		Itt szintén az \texttt{f()} függvény előbb van.
	\end{example}
	\begin{example}
		A linkelés nem változatathat a programon?
		
		main.cpp
		\begin{lstlisting}
std::ostream& o = std::cout << "Hello";
int main() {}
		\end{lstlisting}
		fecske.cpp
		\begin{lstlisting}
std::ostream& o = std::cout << " World";
		\end{lstlisting}
		Itt nem specifikált a két globális változók inicializációs sorrendje, és ha más sorredben fordítjuk, mást ír ki.
		
		{\centering \texttt{g++ main.cpp fecske.cpp \quad $\not=$\quad g++ fecske.cpp main.cpp} \par}
	\end{example}
	\begin{note}
		Ez utolsó példa nem jó kódnak számít, mert nem specifikált viselkedést használ ki.
	\end{note}
\end{document}
