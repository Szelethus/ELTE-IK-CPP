\documentclass[a4paper,11.5pt]{article}
\usepackage[textwidth=170mm, textheight=230mm, inner=20mm, top=20mm, bottom=30mm]{geometry}
\usepackage[normalem]{ulem}
\usepackage[utf8]{inputenc}
\usepackage[T1]{fontenc}
\PassOptionsToPackage{defaults=hu-min}{magyar.ldf}
\usepackage[magyar]{babel}
\usepackage{amsmath, xcolor, amsthm,amssymb,paralist,array, ellipsis, graphicx}
%\usepackage{marvosym}

\usepackage{listings}
\lstset{
	language=C++, 
	basicstyle=\ttfamily, 
	keywordstyle=\color{blue}\ttfamily, 
	stringstyle=\color{red}\ttfamily,
	tabsize = 4
}

\makeatletter
\renewcommand*{\mathellipsis}{%
	\mathinner{%
		\kern\ellipsisbeforegap%
		{\ldotp}\kern\ellipsisgap%
		{\ldotp}\kern\ellipsisgap%
		{\ldotp}\kern\ellipsisaftergap%
	}%
}
\renewcommand*{\dotsb@}{%
	\mathinner{%
		\kern\ellipsisbeforegap%
		{\cdotp}\kern\ellipsisgap%
		{\cdotp}\kern\ellipsisgap%
		{\cdotp}\kern\ellipsisaftergap%
	}%
}
\renewcommand*{\@cdots}{%
	\mathinner{%
		\kern\ellipsisbeforegap%
		{\cdotp}\kern\ellipsisgap%
		{\cdotp}\kern\ellipsisgap%
		{\cdotp}\kern\ellipsisaftergap%
	}%
}
\renewcommand*{\ellipsis@default}{%
	\ellipsis@before
	\kern\ellipsisbeforegap
	.\kern\ellipsisgap
	.\kern\ellipsisgap
	.\kern\ellipsisgap
	\ellipsis@after\relax}
\renewcommand*{\ellipsis@centered}{%
	\ellipsis@before
	\kern\ellipsisbeforegap
	.\kern\ellipsisgap
	.\kern\ellipsisgap
	.\kern\ellipsisaftergap
	\ellipsis@after\relax}
\AtBeginDocument{%
	\DeclareRobustCommand*{\dots}{%
		\ifmmode\@xp\mdots@\else\@xp\textellipsis\fi}}
\def\ellipsisgap{.1em}
\def\ellipsisbeforegap{.05em}
\def\ellipsisaftergap{.05em}
\makeatother

\usepackage{hyperref}
\hypersetup{
	colorlinks = true	
}

\begin{document}
	%%%%%%%%%%%RÖVIDÍTÉSEK%%%%%%%%%%
	\setlength\parindent{0pt}
	\def\s{\hspace{0.2mm}\vphantom{\beta}}
	\def\Z{\mathbb{Z}}
	\def\Q{\mathbb{Q}}
	\def\R{\mathbb{R}}
	\def\C{\mathbb{C}}
	\def\N{\mathbb{N}}
	\def\Ra{\overline{\mathbb{R}}}
	
	\def\sume{\displaystyle\sum_{n=1}^{+\infty}}
	\def\sumn{\displaystyle\sum_{n=0}^{+\infty}}
	
	\def\narrow{\underset{n\rightarrow+\infty}{\longrightarrow}}
	\def\limn{\displaystyle\lim_{n\to +\infty}}
	\def\limx{\displaystyle\lim_{x\to +\infty}}
	
	\theoremstyle{definition}
	\newtheorem{theorem}{Tétel}[subsection] 
	
	\theoremstyle{definition}
	\newtheorem{definition}[theorem]{Definíció} 
	\newtheorem{example}[theorem]{Példa} 
	\newtheorem{task}[theorem]{Feladat} 
	\newtheorem{note}[theorem]{Megjegyzés}
	%%%%%%%%%%%%%%%%%%%%%%%%%%%%%%%%%%%%%%%%%%%%%%%%%%%%%%%%%%%%%%%%%%%%%
	\begin{center}
		{\LARGE\textbf{C++}}
		
		{\Large Gyakorlat jegyzet}
		
		11. óra.
	\end{center}
	A jegyzetet \textsc{Brunner} Tibor készítette \textsc{Horváth} Gábor  előadásán. (\today)
	\section{Objektum orientált programozás}
	Objektum orientált programozásnál e hármat várjuk el:
	\begin{itemize}
		\item enkapszuláció
		\item kód újrafelhasználás
		\item adat elrejtés elve
	\end{itemize}
	Az, hogy egy adott nyelv ez hogyan valósítja meg, az a saját maga dolga. Vannak nyelvek, melyek jobban alkalmasan objektum orientált programozásra, mint mások. Példaképp, a C nyelvre is mondhatjuk, hogy objektum orientált, bár nincs benne örökölődés például, és nem igazán kézenfekvő az adatok elrejtése sem. Nyilvánvalóan ennél egyel praktikusabb nyelvek is léteznek, példaképp Java, és nyilván a c++ is, melyek sokkal erősebb nyelvi eszközökkel rendelkeznek ezek megvalósítására.
	
	Fontos, hogy az objektum orientáltság fogalmát ne csak az osztályokhoz kössük!
	
	Volt már szó \texttt{public} és \texttt{private} kulcsszavakról, ha ezek nem világosak, a 6. gyakorlaton volt róla részletesebben szó.
	
	\medskip
	Tekintsük az alábbi kódot:
	\begin{lstlisting}
class Sikidom
{
	int x;
	int y;
public:
	Sikidom(int x, int y) : x(x), y(y) {}
	double terulet() const
	{
		//...
	}
};
	\end{lstlisting}
	Itt a célunk egy általános síkidom megadása, azonban bajban vagyunk, hisz nem tudjuk megmondani, hogy épp egy tetszőleges síkidomnak milyen lesz a területe. Próbáljuk ehhez hazsnálni az öröklődést: vegyünk át mindent, ami a Síkidomban van, és hozzunk létre ey kör típust!
	
	\begin{lstlisting}
class Kor : public Sikidom
{
	double r;
public:
	double terulet() const
	{
		return r * r * 3.14;
	}
};
	\end{lstlisting}
	A \texttt{Kor} osztály \texttt{Sikidom} összes adattagját és metudást átveszi. Mondhatjuk azt is, hogy a \texttt{Kor} \textit{az egy} \texttt{Sikidom}.
	\begin{note}
		Nagyon fontos, hogy a 3,14-nél pontot és nem vesszpt használunk. Ha ezt írnánk:
		
		{\centering \texttt{int i = 1, 2, f(), g();} \par}
		
		Akkor rendre a vessző paraméter előtti tag mindig kiértékelődik, és utána értékeli ki a következőt. A végén az egész azzal lesz ekvivalens, mintha simán azt írnánk, hogy \texttt{int i = g();}
		
		Egy fél mondat erejéig jegyezzük meg azt is, hogy lebegőpontos számokat sose hasonlítsunk össze! Jusson eszünkbe, ami Numerikus Módszereken hangzik el: 
		%TODO d=0, d+=0,1, 1ig
	\end{note}
	Az, hogy publikusan örököltem, azt jelenti, hogy egy {altípus}-t (\textit{subtype}) hoztunk létre. 
	\begin{lstlisting}
Kor k;
Sikidom s;
s = k; //ok
	\end{lstlisting}
	Ezt azt is jelenti, hogy minden helyre, ahol \texttt{Sikodom}-ot szeretnénk hazsnálni, \texttt{Kor}-t is használhatunk, mert a \texttt{Kor} \texttt{Sikidom} altípusa.
	
	A publikus örökölés azt jelenti, hogy minden, ami publikus volt a bázisosztályban az publikus marad, minden ami protected protected marad, minden ami privát, privát marad.
	
	Ezzel máris megállapíthatjuk, hogy egy picit defektes a küdunk, hisz mi \texttt{Kor}-t azért hoztuk létre, hogy speciálisabb feladatokat tudjuk végrehajtani \texttt{Sikidom} tulajdonságaival, azonban \texttt{x} és \texttt{y} privát, és csak \texttt{Sikidom} tud hozzáférni. Írjuk át ezt protectedre:
\begin{lstlisting}
class Sikidom
{
protected:
	int x;
	int y;
public:
	Sikidom(int x, int y) : x(x), y(y) {}
	double terulet() const {}
};
\end{lstlisting}
	A protected tagokhoz csak az adott osztály fér hozzá, és azon osztályok, melyek ebből örökölnek. Nem csak publikusan tudunk örökölni, protected-del, és private-el is. Az alábbi táblázat mutatja, hogy a bázisosztályban adott védettség
	%TODO finish this
	
	Picit térjünk visza a fenti értékadáshoz, mert ott azért van még mit megbeszélni.
	\begin{lstlisting}
Kor k;
Sikidom s;
s = k; //ok
	\end{lstlisting}
	\texttt{Kor}-ben több információ van, mint \texttt{Sikidom}-ban. AMikor értékül adjuk, hogyan tudjuk egy kisebb, \texttt{Sikidom}-nyi területbe belezsőfolni \texttt{Kor}-t? Sehogy. Itt egy un. \textit{slicing} fog történni, és minden, ami \texttt{Kor}-re specifikus (pl a sugár), az elveszik. Gyakorlatilag ez történik (most tekintsünk el attól, hogy ezek az adattagok védettek):
	\begin{lstlisting}
s.x = k.x;
s.y = k.y;
	\end{lstlisting}
	
	Ezt a \textit{slicing} dolgot meg tudjuk kerülni, ha egy pointert hozunk létre: Hisz egy \texttt{Sikidom}-ra mutató pointer mutathat \texttt{Kor}-re is, és mivel az ''csak'' mutogat, nem fog levágni semmit. 
	\begin{lstlisting}
Kor k;
Sikidom *sp = &k;
	\end{lstlisting}
	Bár ettől még igaz, hogy \texttt{sp}-n keresztül nem fogunk tudni hivatkozni olyan dolgokra, melyek \texttt{Sikidom}-ban nincsenek benne.
	
	Mi fog történni, ha lekérdezzük a területét?
	\begin{lstlisting}
Kor k;
Sikidom *sp = &k;
sp->terulet();
	\end{lstlisting}
	Itt azt fogjuk tapasztalni, hogy \texttt{Sikidom} függvénye fog megíhvódni. Ennek oka az, hogy \texttt{sp} statikus típusa (\textit{static type}) az \texttt{Sikidom}, így hatékonyság végett mindig a statikus típushoz tartozó függvényt hívja meg. Egy pointernek van un. dinamikus típusa is (\textit{dynamic type}), ami ebben az esetben \texttt{Kor} lesz. A dinamikus típust csak futaási időben lehet ismert, hisz azt megállapítani fordítási időben, hogy miylen típusú objektumra fog mutanti az \texttt{sp} pointer, lehetetlen megállapítani. Ha mindig a dinamikus típus szerint hívná meg mindig a nyelv függvényeket, annak nagy futási idejű költsége lenne. Ahhoz, hogy rákényszerítsük a fordítót arra, hogy a dinamikus típus szerint fordítson, használnunk kell a \texttt{virtual} kulcsszót.
	%TODO miért szar a láncolt lista
\begin{lstlisting}
class Sikidom
{
protected:
	int x;
	int y;
public:
	Sikidom(int x, int y) : x(x), y(y) {}
	virtual double terulet() const
	{
		//...
	}
};

class Kor : public Sikidom
{
	double r;
public:
	double terulet() const
	{
		return r * r * 3.14;
	}
};
\end{lstlisting}
	Picit bánthat minket az, hogy hogyan jöhet elő ez a probléma, ugyanis picit úgy tánik, mintha sértenénk a \textit{one definition rule}-t: itt felüldefiniálás fog bekövetkezni, azaz a \texttt{Kor} leárnyákolja a \texttt{Sikidom} \texttt{terulet} függvényét. Viszont ez a leárnyékolás nem ér sokat, ha nincs \texttt{virtual} kulcsszó, és ha egy \texttt{Sikidom} típusú pointerrel akarjuk ezt használni.
	
	Próbáljunk megírni egy sítábor programot, melyben azt oldjuk meg, hogy a kétszobás égyakban vagy 2 fiú, vagy 2 lány aludjon!
	
	\begin{lstlisting}
class Sielo
{
public:
	virtual void szobatars(Sielo*)
	{
		std::cout << "sielo";
	}
};

class Lany : public Sielo
{
public:
	void szobatars(Sielo*)
	{
		std::cout << "lany sielo";
	}
	void szobatars(Lany*)
	{
		std::cout << "lany";
	}
};

class Fiu : public Sielo
{
public:
	void szobatars(Sielo*)
	{
		std::cout << "fiu sielo";
	}
	void szobatars(Fiu*)
	{
		std::cout << "fiu";
	}
};
	\end{lstlisting}
	Egy pillanatra menjünk vissza a körös példára is:
	\begin{lstlisting}
void f(Kor* k)
{
	//...
}
	\end{lstlisting}
	Lehet, az implementáció egy pontján úgy döntünk, hogy ez várjon inkább síkidomokat.
	\begin{lstlisting}
void f(Sikidom* k)
{
	//...
}
	\end{lstlisting}
	Ezt megtehetem, mert mi továbbra is köröket fogunk megadni, és minden kör \textit{az egy} síkidom. Ennek a tanulsága az, hogy paramétereknél \textbf{felfelé} az öröklődési láncon lehet haladni, lefelé nem, hiszen egy síkidomra mutató pointer mutathat körre, de fordítva nem. Azonban a visszatérési értéknél ha:
	\begin{lstlisting}
Sikidom f(...) {}
	\end{lstlisting}
	Úgy döntünk, inkább kört adunk vissza.
	\begin{lstlisting}
Kor f(...) {}
	\end{lstlisting}
	Ezzek semmi gond nem lesz, hisz itt is, minden kör \textit{az egy} síkidom.
	\begin{lstlisting}
Sikidom * sp = f(...);
	\end{lstlisting}
	%TODO ... paraméterek
	\medskip
	
	Visszatérvén a síelőkre:
	\begin{lstlisting}
Lany * l, *l2;
Fiu *f;
std::cout << l->szobatars(l2) << std::endl; //lany

Sielo *s = l;
std::cout << s->szobatars(f) << std::endl; //lany sielo
	\end{lstlisting} 
	Itt sikerült elírni, hogy a fiúnak lány legyen a szobatársa. Figyeljük meg, hogy két különböző paraméterű \texttt{szobatars} metódust írtunk, de csak az virtuális, mely egy \texttt{Sielo}-t vár. Ezért, mivel minden \texttt{Fiu} \textit{az egy} \texttt{Sielo}, teljesen szabályos lesz a \texttt{szobatard(Sielo*)} függvényhívás.
	
	%TODO jobb struktúrálás
\end{document}